% latex uft-8
\documentclass[uplatex,a4paper,11pt,oneside,openany]{jsarticle}
%
\usepackage[T1]{fontenc}
\usepackage[dvipdfmx]{graphicx}
\usepackage[dvipdfmx]{color}
\usepackage{amsmath,amssymb}
\usepackage{enumerate}
\usepackage{bm}
\usepackage{graphicx}
\usepackage{ascmac}
\usepackage{setspace}
\usepackage{here}
\usepackage{url}
\usepackage{ulem}
\usepackage{colortbl}
\usepackage{comment}
\usepackage{setspace}
\usepackage{multicol}
\usepackage{multicolpar}
\usepackage{listings,jlisting} %日本語のコメントアウトをする場合jlistingが必要
%ここからソースコードの表示に関する設定
\lstset{
	%プログラム言語(複数の言語に対応,C,C++も可)
	language = Python,
	%背景色と透過度
	backgroundcolor = {\color[gray]{.95}},
	%枠外に行った時の自動改行
	breaklines = true,
	%自動改行後のインデント量(デフォルトでは20[pt])
	breakindent = 10pt,
	%標準の書体
	%basicstyle = \ttfamily\scriptsize,
	basicstyle = \fontsize{8}{10}\selectfont\ttfamily,
	%コメントの書体
	commentstyle = {\itshape \color[cmyk]{1,0.4,1,0}},
	%関数名等の色の設定
	classoffset = 0,
	%キーワード(int, ifなど)の書体
	keywordstyle = {\bfseries \color[cmyk]{0,1,0,0}},
	%表示する文字の書体
	stringstyle = {\ttfamily \color[rgb]{0,0,1}},
	%枠 "t"は上に線を記載,"T"は上に二重線を記載
	%他オプション:leftline,topline,bottomline,lines,single,shadowbox
	frame = TBrl,
	%frameまでの間隔(行番号とプログラムの間)
	framesep = 5pt,
	%行番号の位置
	numbers = left,
	%行番号の間隔
	stepnumber = 1,
	%行番号の書体
	numberstyle = \tiny,
	%タブの大きさ
	tabsize = 4,
	%キャプションの場所("tb"ならば上下両方に記載)
	captionpos = t
}
%ここまでソースコードの表示に関する設定

\begin{document}
	\title{C言語:三目並べ(TicTacToe)}
	\author{○×工業高校 機械工学科 2年}
	\date{2023年10月11日}
	\maketitle
	\pagestyle{empty}

\newpage
	
\section{ゲームの概要}

プログラムを実行すると、盤面が表示され、
×の石を置く場所を指定するよう促されます。

画面上に示された番号を入力すると、
その番号のスロットに×の石が置かれた盤面が表示され、
次の手番の○に、石を置く場所を指定するように促されます。

手番を交互に変えながらゲームは進み、縦、横、斜めの何れかに、
先に一列に自分の石を並べた方が勝ちとなります。

既に石の置かれているスロット番号を指定することはできません。
また、スロット番号として 0 〜 8 以外の数値を指定することもできません。

\begin{spacing}{0.74}
	\begin{verbatim}
		スタート! [Tic Tac Toe]
		/---|---|---\
		| 0 | 1 | 2 |
		|---|---|---|
		| 3 | 4 | 5 |
		|---|---|---|
		| 6 | 7 | 8 |
		\---|---|---/
		'X' さんのturnです
		石を置く場所 0 〜 8 を指定して下さい : 4
		/---|---|---\
		| 0 | 1 | 2 |
		|---|---|---|
		| 3 | X | 5 |
		|---|---|---|
		| 6 | 7 | 8 |
		\---|---|---/
		'O' さんのturnです
		石を置く場所 0 〜 8 を指定して下さい : 2
		/---|---|---\
		| 0 | 1 | O |
		|---|---|---|
		| 3 | X | 5 |
		|---|---|---|
		| 6 | 7 | 8 |
		\---|---|---/
		'X' さんのturnです
		石を置く場所 0 〜 8 を指定して下さい :
	\end{verbatim}
\end{spacing}

\newpage

\section{主処理(main)}

\begin{lstlisting}[]
	int main(int argc, char *argv[]){
		/* 先手後手を決定 */
		int turn = BATSU,  winner, num;
		if (1 < argc){
			if (!strcmp(argv[1], "-r"))
			turn = MARU;
		}
		printf("スタート! [Tic Tac Toe]\n");
		do{
			printBoard();            /* ①盤面の表示 */
			num = slotNum(turn); /* ②手を入力 */
			board[num] = turn;       /* ③手を盤面に配置 */
			turn = switchTurn(turn); /* ④手番の交代 */
			winner = checkWinner();  /* ⑤勝敗の判定 */
		} while (winner == NEXT);
		/* 対戦結果の表示 */
		printBoard();
		result(winner);
		return 0;
	}
\end{lstlisting}

\section{盤面の表示(printBoard)}

\begin{lstlisting}
	void printBoard() {
		char bd[9];
		int i;
		for (i = 0; i < 9; i++) {
			if (board[i] == MARU) bd[i] = 'O';
			else if (board[i] == BATSU) bd[i] = 'X';
			else bd[i] = '0' + i;
		}
		printf("\n/---|---|---\\\n");
		printf("| %c | %c | %c |\n", bd[0], bd[1], bd[2]);
		printf("|---|---|---|\n");
		printf("| %c | %c | %c |\n", bd[3], bd[4], bd[5]);
		printf("|---|---|---|\n");
		printf("| %c | %c | %c |\n", bd[6], bd[7], bd[8]);
		printf("\\---|---|---/\n");
	}
\end{lstlisting}

\section{手番の交代(switchTurn)}

\begin{lstlisting}
	int switchTurn(int turn) {
		if (turn== BATSU)return MARU;
		return BATSU;
	}
\end{lstlisting}

\section{入力(slotNum)}

\begin{lstlisting}
	int slotNum(int turn) {
		int num;
		char *fig = "";
		if (turn==MARU) fig = "'O'";
		else if (turn==BATSU) fig = "'X'";
		do {
			printf("\n %s さんのturnです\n石を置く場所 0 〜 8 を指定して下さい:", fig);
			//while (getchar() != '\n');  /* 標準入力バッファのクリア */
			scanf("%d", &num);
			if (!(0 <= num && num < 9)) {
				printf("再指定:0 〜 8 を指定して下さい");
				continue;
			}
			if (board[num] != num) {
				printf("再指定:そこには既に石が置かれています\n");
				continue;
			}
			break;
		} while (1);
		return num;
	}
\end{lstlisting}

\section{判定(checkWinner)}

\begin{lstlisting}
	int lineSum(int n1, int n2, int n3) {
		return board[n1] + board[n2] + board[n3];
	}
	int checkWinner() {
		int i, line = 0;
		for (i = 0; i < 8; i++) {
			switch (i) {
				case 0: line = lineSum(0, 1, 2); break;
				case 1: line = lineSum(3, 4, 5); break;
				case 2: line = lineSum(6, 7, 8); break;
				case 3: line = lineSum(0, 3, 6); break;
				case 4: line = lineSum(1, 4, 7); break;
				case 5: line = lineSum(2, 5, 8); break;
				case 6: line = lineSum(0, 4, 8); break;
				case 7: line = lineSum(2, 4, 6); break;
			}
			if (line == 3 * MARU) return MARU;
			else if (line == 3 * BATSU) return BATSU;
		}
		for (i = 0; i < 9; i++){
			if (0 <= board[i] && board[i] < 9) return NEXT;
		}
		return DRAW;
	}
\end{lstlisting}

\newpage

\section{結果表示(result)}

\begin{lstlisting}
	void result(int winner) {
		printf("\n");
		switch (winner) {
			case DRAW: printf("引き分け\t"); break;
			case MARU: printf("'O' の勝ち\t"); break;
			case BATSU: printf("'X' の勝ち\t"); break;
		}
		printf("またね!\n");
	}
\end{lstlisting}

\section{各種宣言など}

これは冒頭に記述する

\begin{lstlisting}
	#include <stdio.h>
	#include <string.h>
	
	// function prototypes
	/*
	int lineSum(int, int, int);
	int switchTurn(int);
	void printBoard();
	int slotNum(int);
	int checkWinner();
	void result(int);
	*/
	
	static int board[] = {0, 1, 2, 3, 4, 5, 6, 7, 8};
	#define MARU 10
	#define BATSU -10
	#define DRAW 100
	#define NEXT 200
\end{lstlisting}

\newpage

\section{プログラムの全体}

\lstinputlisting[caption=Tic Tac Toe(三目並べ),label=tictactoe]{../src/tictactoe.c}

%
\begin{thebibliography}{99}
	\bibitem{1} 田中賢一郎、「ゲームで学ぶ Java Script 入門」、インプレス
\end{thebibliography}
%
\end{document}