% latex uft-8
\documentclass[uplatex,a4paper,11pt,oneside,openany]{jsarticle}
%
\usepackage[T1]{fontenc}
\usepackage[dvipdfmx]{graphicx}
\usepackage[dvipdfmx]{color}
\usepackage{amsmath,amssymb}
\usepackage{enumerate}
\usepackage{bm}
\usepackage{graphicx}
\usepackage{ascmac}
\usepackage{setspace}
\usepackage{here}
\usepackage{url}
\usepackage{ulem}
\usepackage{colortbl}
\usepackage{comment}
\usepackage{setspace}
\usepackage{multicol}
\usepackage{multicolpar}
\usepackage{listings,jlisting} %日本語のコメントアウトをする場合jlistingが必要
%ここからソースコードの表示に関する設定
\lstset{
	%プログラム言語(複数の言語に対応,C,C++も可)
	language = Python,
	%背景色と透過度
	backgroundcolor = {\color[gray]{.95}},
	%枠外に行った時の自動改行
	breaklines = true,
	%自動改行後のインデント量(デフォルトでは20[pt])
	breakindent = 10pt,
	%標準の書体
	%basicstyle = \ttfamily\scriptsize,
	basicstyle = \fontsize{8}{10}\selectfont\ttfamily,
	%コメントの書体
	commentstyle = {\itshape \color[cmyk]{1,0.4,1,0}},
	%関数名等の色の設定
	classoffset = 0,
	%キーワード(int, ifなど)の書体
	keywordstyle = {\bfseries \color[cmyk]{0,1,0,0}},
	%表示する文字の書体
	stringstyle = {\ttfamily \color[rgb]{0,0,1}},
	%枠 "t"は上に線を記載,"T"は上に二重線を記載
	%他オプション:leftline,topline,bottomline,lines,single,shadowbox
	frame = TBrl,
	%frameまでの間隔(行番号とプログラムの間)
	framesep = 5pt,
	%行番号の位置
	numbers = left,
	%行番号の間隔
	stepnumber = 1,
	%行番号の書体
	numberstyle = \tiny,
	%タブの大きさ
	tabsize = 4,
	%キャプションの場所("tb"ならば上下両方に記載)
	captionpos = t
}
%ここまでソースコードの表示に関する設定

\begin{document}
	\title{C言語:スライド・パズル(15Puzzle)}
	\author{○×工業高校 機械工学科 2年}
	\date{2023年10月11日}
	\maketitle
	\pagestyle{empty}

\newpage

\section{ゲームの概要}

$4 \times 4$に区切った盤面上の各タイルに、$0〜15$の番号が割り振られている。
0 が割り振られたタイルは空欄になっていて、他のタイルはその空欄にスライドさせて
移動することができる。
最初は不規則に並べられている盤面ですが、空欄の上下、あるいは空欄の左右のタイルを
選んでは、空欄の方向にスライドさせる事によって、最終的に1から15まで規則正しく
並んだ盤面状態を目指すゲームである。

\begin{spacing}{0.74}
	\begin{verbatim}
		% ./slidetile
		
		[11][ 3][ 6][10]
		[ 9][ 1][ 8][14]
		[ 2][ 4][ 7][12]
		[13][15][  ][ 5]
		
		Select number:6
		
		[11][ 3][  ][10]
		[ 9][ 1][ 6][14]
		[ 2][ 4][ 8][12]
		[13][15][ 7][ 5]
		
		Select number:11
		
		[  ][11][ 3][10]
		[ 9][ 1][ 6][14]
		[ 2][ 4][ 8][12]
		[13][15][ 7][ 5]
		
		Select number:11
		
		[11][  ][ 3][10]
		[ 9][ 1][ 6][14]
		[ 2][ 4][ 8][12]
		[13][15][ 7][ 5]
		
		Select number:
	\end{verbatim}
\end{spacing}

\newpage

\section{主処理(main)}

\begin{lstlisting}
	int main(void) {
		int sel;
		init();
		do {
			disp();
			do {
				printf("\nSelect number:");
				scanf("%d", &sel);
			} while( sel>ROW*COLUMN );
			moveTile(sel);
		} while( check() );
		
		return 0;
	}
\end{lstlisting}

\section{盤面の初期化(init)}

乱数を使ってタイルをシャフルするので、
時刻を乱数の種に指定することによって、
プログラムを起動する度に、異なるタイル配置になる様にしている

\begin{lstlisting}
	void init(){
		int i, j, n;
		srand((unsigned)time(NULL));
		for (i = 0; i < ROW; i++){
			for (j = 0; j < COLUMN; j++){
				n = i * COLUMN + j;
				tiles[n].num = n;
				tiles[n].row = i;
				tiles[n].clm = j;
			}
		}
		shuffle(1000);
	}
\end{lstlisting}

\section{シャッフル(shuffle)}

乱数を使って、タイルを不規則に選択し移動させている

\begin{lstlisting}
	void shuffle(int n){
		int sel, min = 0, max = ROW*COLUMN;
		do{
			sel = (rand() % (max - min + 1)) + min;
			moveTile(sel);
			n--;
		} while (0 < n);
	}
\end{lstlisting}

\newpage

\section{盤面の表示(disp)}

\begin{lstlisting}
	void disp(){
		int i, j, n;
		printf("\n");
		for (i = 0; i < ROW; i++){
			for (j = 0; j < COLUMN; j++){
				n = i * COLUMN + j;
				if (tiles[n].num == 0){
					printf("[  ]");
				} else {
					printf("[%2d]", tiles[i * COLUMN + j].num);
				}
			}
			printf("\n");
		}
	}
\end{lstlisting}

\section{タイルのスライド(moveTile)}

\begin{lstlisting}
	int findTileNum(int num){
		int i, j;
		for (i = 0; i < ROW * COLUMN; i++){
			if (tiles[i].num == num){
				return i;
			}
		}
		return -1;
	}
	void swapTile(int n1, int n2){
		int tmp = tiles[n1].num; 
		tiles[n1].num = tiles[n2].num;
		tiles[n2].num = tmp;
	}
	void swapTileR(int c, int r1, int r2){
		int n1 = r1 * COLUMN + c;
		int n2 = r2 * COLUMN + c;
		swapTile(n1, n2);
	}
	void swapTileC(int r, int c1, int c2){
		int n1 = r * COLUMN + c1;
		int n2 = r * COLUMN + c2;
		swapTile(n1, n2);
	}
	void moveTile(int sel){
		int i, j;
		int s = findTileNum(sel);
		int sr = tiles[s].row;
		int sc = tiles[s].clm;
		int z = findTileNum(0);
		int zr = tiles[z].row;
		int zc = tiles[z].clm;
		if (sr == zr){
			if (sc < zc){
				for (j = zc; j > sc; j--){
					if (0 <= j - 1){
						swapTileC(sr, j, j - 1);
					}
				}
			} else if (zc < sc) {
				for (j = zc; j < sc; j++){
					if (j + 1 < COLUMN){
						swapTileC(sr, j, j + 1);
					}
				}
			}
		}
		if (sc == zc){
			if (sr < zr){
				for (j = zr; j > sr; j--){
					if (0 <= j - 1){
						swapTileR(sc, j, j - 1);
					}
				}
			} else if (zr < sr){
				for (j = zr; j < sr; j++){
					if (j + 1 < ROW){
						swapTileR(sc, j, j + 1);
					}
				}
			}
		}
	}
\end{lstlisting}

\section{完成チェック(check)}

\begin{lstlisting}
	enum BOOLEAN check(){
		int i, j, n;
		for (i = 0; i < ROW; i++){
			for (j = 0; j < COLUMN; j++){
				n = i * COLUMN + j;
				if (tiles[n].num != n){
					return true;
				}
			}
		}
		return false;
	}
\end{lstlisting}

\newpage

\section{各種宣言など}

これは冒頭に記述する

\begin{lstlisting}
	#include <stdio.h>
	#include <stdlib.h>
	#include <time.h>
	
	// 定数
	#define ROW 4
	#define COLUMN 4
	// 構造体
	struct Tile{
		int num;
		int row;
		int clm;
	};
	enum BOOLEAN{
		false, /* false = 0, true = 1 */
		true
	};
	// function prototypes
	/*
	int findTileNum(int);
	void swapTile(int,int);
	void swapTileR(int,int,int);
	void swapTileC(int,int,int);
	void moveTile(int);
	void shuffle(int);
	enum BOOLEAN check();
	void disp();
	void init();
	*/
	
	static struct Tile tiles[16];   /* 16 = ROW * COLUMN */
\end{lstlisting}

\newpage

\section{プログラムの全体}

\lstinputlisting[caption=スライド・パズル,label=slidetile]{../src/slidetile.c}

\end{document}