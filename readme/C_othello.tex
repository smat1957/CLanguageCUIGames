% latex uft-8
\documentclass[uplatex,a4paper,11pt,oneside,openany]{jsarticle}
%
\usepackage[T1]{fontenc}
\usepackage[dvipdfmx]{graphicx}
\usepackage[dvipdfmx]{color}
\usepackage{amsmath,amssymb}
\usepackage{enumerate}
\usepackage{bm}
\usepackage{graphicx}
\usepackage{ascmac}
\usepackage{setspace}
\usepackage{here}
\usepackage{url}
\usepackage{ulem}
\usepackage{colortbl}
\usepackage{comment}
\usepackage{setspace}
\usepackage{multicol}
\usepackage{multicolpar}
\usepackage{listings,jlisting} %日本語のコメントアウトをする場合jlistingが必要
%ここからソースコードの表示に関する設定
\lstset{
	%プログラム言語(複数の言語に対応,C,C++も可)
	language = Python,
	%背景色と透過度
	backgroundcolor = {\color[gray]{.95}},
	%枠外に行った時の自動改行
	breaklines = true,
	%自動改行後のインデント量(デフォルトでは20[pt])
	breakindent = 10pt,
	%標準の書体
	%basicstyle = \ttfamily\scriptsize,
	basicstyle = \fontsize{8}{10}\selectfont\ttfamily,
	%コメントの書体
	commentstyle = {\itshape \color[cmyk]{1,0.4,1,0}},
	%関数名等の色の設定
	classoffset = 0,
	%キーワード(int, ifなど)の書体
	keywordstyle = {\bfseries \color[cmyk]{0,1,0,0}},
	%表示する文字の書体
	stringstyle = {\ttfamily \color[rgb]{0,0,1}},
	%枠 "t"は上に線を記載,"T"は上に二重線を記載
	%他オプション:leftline,topline,bottomline,lines,single,shadowbox
	frame = TBrl,
	%frameまでの間隔(行番号とプログラムの間)
	framesep = 5pt,
	%行番号の位置
	numbers = left,
	%行番号の間隔
	stepnumber = 1,
	%行番号の書体
	numberstyle = \tiny,
	%タブの大きさ
	tabsize = 4,
	%キャプションの場所("tb"ならば上下両方に記載)
	captionpos = t
}
%ここまでソースコードの表示に関する設定

\begin{document}
	\title{C言語:オセロ(リバーシ)}
	\author{◯×工業高校 機械工学科 2年}
	\date{\today}
	\maketitle
	\pagestyle{empty}

%\newpage
	
\section{ゲームの概要}

\includegraphics[keepaspectratio,scale=0.5]{othelloview.png}

\section{主処理}

ゲーム盤のサイズは、$4 \times 4, 6 \times 6, 8 \times 8$など偶数の幅を $\#define$ 文で設定する。
行番号(1,2,3,...)と列記号文字(a,b,c,...)の連続する半角2文字で石の位置を指定する。
手番の石を置けないところを指定しても無視される。

\begin{lstlisting}
	POS decode(char* str){
		POS pos;
		pos.x = atoi(str)-1;
		char* alpha="abcdefghijklmnopqrstuvwxyz";
		int i;
		for(i=0; i<BOARDW; i++){
			if(*(str+1)==alpha[i])
			break;
		}
		pos.y = i;
		return pos;
	}
	
	void event(POS pos){
		if(endflag){
			initboard();
			return;
		}
		if(0<passcount){
			nextturn();
			drawboard();
			return;
		}
		if(!isinside(pos))
		return;
		if(0==flippable(pos, turn))
		return;
		for(int i=0; i<8; i++){
			int loop = search(pos, i, turn);
			POS temp = pos;
			for(int j=0; j<loop; j++){
				temp = movepos(temp, i);
				setstone(temp, turn);
			}
		}
		setstone(pos, turn);
		nextturn();
		drawboard();
	}
	
	int main(void){
		if(!initboard())
		return 8;
		char inpt[]="   ";
		while(!endflag){
			printf("=>");
			scanf("%s", inpt);
			POS pos = decode(inpt);
			printf("\n");
			event(pos);
		}
		return 0;
	}
\end{lstlisting}

\section{盤面の表示}

盤面を表示するたびに、石の数を数えて表示している。

\begin{lstlisting}
	int count(int color){
		int num=0;
		for(int i=0; i<(BOARDW * BOARDW); i++)
		if(board[i]==color)
		num++;
		return num;
	}
	
	void drawboard(){
		for(int x=0; x<BOARDW; x++){
			if(x==0){
				printf("    ");
				char a='a';
				for(int i=0; i<BOARDW; i++)
				printf("%c ", a+i);
				printf("\n");
			}
			printf("%2d  ", x+1);
			for(int y=0; y<BOARDW; y++){
				int index = y * BOARDW + x;
				switch(board[index]){
					case BLACK:printf("%s ", TILE[BLACK]);break;
					case WHITE:printf("%s ", TILE[WHITE]);break;
					case NONE: printf("%s ", TILE[NONE]); break;
				}
			}
			printf("\n");
		}
		printf("\n%s(%2d), %s(%2d)",TILE[BLACK],count(BLACK),TILE[WHITE],count(WHITE));
		if(!endflag)
		printf(" : turn(%s)\n", TILE[turn]);
		else
		printf("\n");
	}
\end{lstlisting}

\section{初期化}

ボードの幅が偶数でないとゲームを始められない。

\begin{lstlisting}
	enum BOOLEAN initboard(){
		if(BOARDW%2)
			return false;
		int x, y;
		POS pos;
		for(y=0; y<BOARDW; y++)
			for(x=0; x<BOARDW; x++){
				pos.x = x; pos.y = y;
				setstone(pos, NONE);
			}
		pos.x = pos.y = BOARDW/2-1;
		setstone(pos, BLACK);
		pos.x = BOARDW/2; pos.y = pos.x-1;
		setstone(pos, WHITE);
		pos.y = BOARDW/2; pos.x = pos.y-1;
		setstone(pos, WHITE);
		pos.x = pos.y = BOARDW/2;
		setstone(pos, BLACK);
		turn = BLACK;
		passcount = 0;
		endflag = false;
		drawboard();
		return true;
	}
\end{lstlisting}

\section{手番の交代}

指定された場所が、盤の内部の位置かどうか、
相手の石を反転させられるかどうかをチェックしている。

\begin{lstlisting}
	POS movepos(POS pos, int v){
		POS p;
		p.x = pos.x + UNITV[v][0];
		p.y = pos.y + UNITV[v][1];
		return p;
	}
	
	enum BOOLEAN isinside(POS pos){
		if( (pos.x<0) || (BOARDW<=pos.x) )
			return false;
		if( (pos.y<0) || (BOARDW<=pos.y) )
			return false;
		return true;
	}
	
	int search(POS pos, int v, int num){
		int piece = 0;
		while(true){
			pos = movepos(pos, v);
			if(!isinside(pos))
				return 0;
			if(getstone(pos)==NONE)
				return 0;
			if(getstone(pos)==num)
				break;
			piece ++;
		}
		return piece;
	}
	
	int flippable(POS pos, int num){
		if(getstone(pos)!=NONE)
			return 0;
		int total = 0;
		int vec[]={0,0};
		for(int i=0; i<8; i++)
			total += search(pos, i, num);
		return total;
	}
	
	void nextturn(){
		turn ^= 1;
		int empty = 0;
		for(int y=0; y<BOARDW; y++)
			for(int x=0; x<BOARDW; x++){
				POS pos; pos.x=x; pos.y=y;
				if(getstone(pos)==NONE)
					empty++;
				if(0<flippable(pos,turn)){
					passcount = 0;
					return;
				}
			}
		if(empty==0){
			endflag = true;
			return;
		}
		passcount++;
		if(2<=passcount)
			endflag = true;
	}
\end{lstlisting}

\section{各種宣言など}

これは冒頭に記述する。

$BOOLEAN$形を定義している。ボードの幅は $\#define$ 文で指定する。

\begin{lstlisting}
	#include <stdio.h>
	#include <stdlib.h>
	
	enum BOOLEAN{
		false,    /* false=0, true=1 */
		true
	};
	
	#define BOARDW (8) // 4, 6, 8, ....
	#define BLACK (0)
	#define WHITE (1)
	#define NONE (2)
	
	int UNITV[][2] = {{0,-1},{1,-1},{1,0},{1,1},{0,1},{-1,1},{-1,0},{-1,-1}};
	int turn = BLACK;
	int passcount = 0;
	int endflag = false;
	int board[BOARDW * BOARDW];
	
	const char* TILE[] = {
		"●",  //TILE_BLACK
		"◯",  //TILE_WHITE
		"."   //TILE_NONE
	};
	
	typedef struct {
		int x, y;
	}POS;
	
	void setstone(POS pos, int num){
		int index = (pos.y * BOARDW) + pos.x;
		board[index] = num;
	}
	
	int getstone(POS pos){
		int index = (pos.y * BOARDW) + pos.x;
		return board[index];
	}
\end{lstlisting}

%\newpage

\section{プログラムの全体}

\lstinputlisting[caption=オセロ,label=othello]{../src/othello.c}


\end{document}