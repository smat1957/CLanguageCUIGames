% latex uft-8
\documentclass[uplatex,a4paper,11pt,oneside,openany]{jsarticle}
%
\usepackage[T1]{fontenc}
\usepackage[dvipdfmx]{graphicx}
\usepackage[dvipdfmx]{color}
\usepackage{amsmath,amssymb}
\usepackage{enumerate}
\usepackage{bm}
\usepackage{graphicx}
\usepackage{ascmac}
\usepackage{setspace}
\usepackage{here}
\usepackage{url}
\usepackage{ulem}
\usepackage{colortbl}
\usepackage{comment}
\usepackage{setspace}
\usepackage{multicol}
\usepackage{multicolpar}
\usepackage{listings,jlisting} %日本語のコメントアウトをする場合jlistingが必要
%ここからソースコードの表示に関する設定
\lstset{
	%プログラム言語(複数の言語に対応,C,C++も可)
	language = Python,
	%背景色と透過度
	backgroundcolor = {\color[gray]{.95}},
	%枠外に行った時の自動改行
	breaklines = true,
	%自動改行後のインデント量(デフォルトでは20[pt])
	breakindent = 10pt,
	%標準の書体
	%basicstyle = \ttfamily\scriptsize,
	basicstyle = \fontsize{8}{10}\selectfont\ttfamily,
	%コメントの書体
	commentstyle = {\itshape \color[cmyk]{1,0.4,1,0}},
	%関数名等の色の設定
	classoffset = 0,
	%キーワード(int, ifなど)の書体
	keywordstyle = {\bfseries \color[cmyk]{0,1,0,0}},
	%表示する文字の書体
	stringstyle = {\ttfamily \color[rgb]{0,0,1}},
	%枠 "t"は上に線を記載,"T"は上に二重線を記載
	%他オプション:leftline,topline,bottomline,lines,single,shadowbox
	frame = TBrl,
	%frameまでの間隔(行番号とプログラムの間)
	framesep = 5pt,
	%行番号の位置
	numbers = left,
	%行番号の間隔
	stepnumber = 1,
	%行番号の書体
	numberstyle = \tiny,
	%タブの大きさ
	tabsize = 4,
	%キャプションの場所("tb"ならば上下両方に記載)
	captionpos = t
}
%ここまでソースコードの表示に関する設定

\begin{document}
	\title{C言語:神経衰弱(Flip Cards)}
	\author{○×工業高校 機械工学科 2年}
	\date{2023年10月11日}
	\maketitle
	\pagestyle{empty}

%\newpage
	
\section{ゲームの概要}

カードを2枚開いて、一致すれば得点となり、不一致なら再度伏せて相手の手番になる。

\begin{spacing}{0.74}
	\begin{verbatim}
		% ./flipcard
		[ a ][ b ][ c ][ d ][ e ]
		[ f ][ g ][ h ][ i ][ j ]
		[ k ][ l ][ m ][ n ][ o ]
		[ p ][ q ][ r ][ s ][ t ]
		
		Aさん:0 点	Bさん:0 点   = Aさんの番です =
		Select 1st card : a
		
		[ 4 ][ b ][ c ][ d ][ e ]
		[ f ][ g ][ h ][ i ][ j ]
		[ k ][ l ][ m ][ n ][ o ]
		[ p ][ q ][ r ][ s ][ t ]
		
		Select 2nd card : p
		
		[ 4 ][ b ][ c ][ d ][ e ]
		[ f ][ g ][ h ][ i ][ j ]
		[ k ][ l ][ m ][ n ][ o ]
		[ 5 ][ q ][ r ][ s ][ t ]
		
		ハズレ
		
		[ a ][ b ][ c ][ d ][ e ]
		[ f ][ g ][ h ][ i ][ j ]
		[ k ][ l ][ m ][ n ][ o ]
		[ p ][ q ][ r ][ s ][ t ]
		
		Aさん:0 点	Bさん:0 点   = Bさんの番です =
		Select 1st card : e
		
		[ a ][ b ][ c ][ d ][ 4 ]
		[ f ][ g ][ h ][ i ][ j ]
		[ k ][ l ][ m ][ n ][ o ]
		[ p ][ q ][ r ][ s ][ t ]
		
		Select 2nd card : a
		
		[ 4 ][ b ][ c ][ d ][ 4 ]
		[ f ][ g ][ h ][ i ][ j ]
		[ k ][ l ][ m ][ n ][ o ]
		[ p ][ q ][ r ][ s ][ t ]
		
		当たり
		
		[ 4 ][ b ][ c ][ d ][ 4 ]
		[ f ][ g ][ h ][ i ][ j ]
		[ k ][ l ][ m ][ n ][ o ]
		[ p ][ q ][ r ][ s ][ t ]
		
		Aさん:0 点	Bさん:1 点   = Bさんの番です =
		Select 1st card : t
		
		[ 4 ][ b ][ c ][ d ][ 4 ]
		[ f ][ g ][ h ][ i ][ j ]
		[ k ][ l ][ m ][ n ][ o ]
		[ p ][ q ][ r ][ s ][ 0 ]
		
		Select 2nd card : q
		
		[ 4 ][ b ][ c ][ d ][ 4 ]
		[ f ][ g ][ h ][ i ][ j ]
		[ k ][ l ][ m ][ n ][ o ]
		[ p ][ 9 ][ r ][ s ][ 0 ]
		
		ハズレ
		
		[ 4 ][ b ][ c ][ d ][ 4 ]
		[ f ][ g ][ h ][ i ][ j ]
		[ k ][ l ][ m ][ n ][ o ]
		[ p ][ q ][ r ][ s ][ t ]
		
		Aさん:0 点	Bさん:1 点   = Aさんの番です =
		Select 1st card : 
	\end{verbatim}
\end{spacing}

\newpage

\section{主処理(main)}

\begin{lstlisting}
	int main( void ){
		int cnum1, cnum2, Closed=ROW*COLUMN;
		int pointA=0, pointB=0;
		init();
		disp();
		do{
			printf("Aさん:%d 点\tBさん:%d 点\n", pointA, pointB);
			if (Turn)
			printf("= Aさんの番です =\n");
			else
			printf("= Bさんの番です =\n");
			cnum1 = getNum("1st");
			cnum2 = getNum("2nd");
			if ( match(cnum1, cnum2) ){
				printf("当たり\n");
				Closed -= 2;
				if (Turn)
				pointA++;
				else
				pointB++;
			} else {
				printf("ハズレ\n");
				closeCard(cnum1);
				closeCard(cnum2);
				Turn = !Turn;   /* 手番の交代 */
			}
			sleep(4);   /* 開いたカードを見せておく時間 */
			disp();
		} while ( 0 < Closed );
		
		return 0;
	}
\end{lstlisting}

\section{盤面の初期化(init)}

プログラム中で乱数を使うので、時刻を乱数の種に指定して、
プログラムを起動するたびに異なるカード配置になるようにしている

\begin{lstlisting}
	void init(){
		srand((unsigned)time(NULL));
		int i, j, num;
		for (i = 0; i < ROW/2; i++){
			for (j = 0; j < COLUMN; j++){
				num = i * COLUMN + j;
				cards[num].num = cards[ROW * COLUMN - num].num = num;
				cards[num].row = cards[ROW * COLUMN - num].row = i;
				cards[num].clm = cards[ROW * COLUMN - num].clm = j;
				cards[num].opn = cards[ROW * COLUMN - num].opn = false;
			}
		}
		shuffle();
	}
\end{lstlisting}

\section{カードのシャフル(shuffle, swapCards)}

乱数を使ってカード配置を入れ替えている

\begin{lstlisting}
	void swapCards(int n1, int n2){
		struct Card temp;
		temp = cards[n2];
		cards[n2] = cards[n1];
		cards[n1] = temp;
	}
	void shuffle(){
		int sel, min = 0, max = ROW*COLUMN;
		while (0 < max){
			sel = (rand() % (max - min + 1)) + min;
			swapCards(sel, max--);
		}
	}
\end{lstlisting}

\section{盤面の表示(disp)}

\begin{lstlisting}
	void disp(){
		int i, j, n;
		printf("\n");
		for (i = 0; i < ROW; i++){
			for (j = 0; j < COLUMN; j++){
				n = i * COLUMN + j;
				if ( !cards[n].opn )
				printf("[ %c ]", 'a' + i * COLUMN + j );
				else
				printf("[ %d ]", cards[i * COLUMN + j].num );
			}
			printf("\n");
		}
		printf("\n");
	}
\end{lstlisting}

\section{入力(getNum)}

\begin{lstlisting}
	int getNum(char* s){
		int cnum;
		char str[24], work[24];
		do {
			sprintf(work, "%s%s%s", "\tSelect ", s, " card : ");
			printf("%s", work);
			scanf("%s", str);
			cnum = str[0] - 'a';
		} while (cnum > ROW * COLUMN);
		openCard(cnum);
		disp();
		return cnum;
	}
\end{lstlisting}

\section{カードの開閉(openCard, closeCard)}

\begin{lstlisting}
	void openCard(int n){
		cards[n].opn = true;
	}
	void closeCard(int n){
		cards[n].opn = false;
	}
\end{lstlisting}

\section{一致不一致の判定(match)}

\begin{lstlisting}
	enum BOOLEAN match(int n1, int n2){
		if ( cards[n1].num == cards[n2].num )
		return true;
		return false;
	}
\end{lstlisting}

\begin{comment}
	\section{手番の交代(switchTurn)}
	
	\begin{lstlisting}
		void switchTurn(enum BOOLEAN t){
			Turn = !t;
		}
	\end{lstlisting}
	
	\section{時間稼ぎ(wasteTime)}
	
	開いたカードを見せておく時間
	
	\begin{lstlisting}
		void wasteTime(int sec){
			sleep(sec);
		}
	\end{lstlisting}
\end{comment}

\section{各種宣言など}

これは冒頭に記述する

\begin{lstlisting}
	#include <stdio.h>
	#include <stdlib.h>
	#include <time.h>
	#include <unistd.h>
	// 定数
	#define ROW 4
	#define COLUMN 5
	// 構造体
	enum BOOLEAN {
		false, /* false = 0, true = 1 */
		true
	};
	struct Card {
		int num;
		int row;
		int clm;
		enum BOOLEAN opn;
	};
	// function prototypes
	/*
	void openCard(int);
	void closeCard(int);
	void switchTurn(enum BOOLEAN);
	void disp();
	int getNum(char*);
	enum BOOLEAN match(int, int);
	void wasteTime(int);
	void swapCards(int,int);
	void shuffle();
	void init();
	*/
	static struct Card cards[20]; /* 20 = ROW * COLUMN */
	enum BOOLEAN Turn = true;  
\end{lstlisting}

\newpage

\section{プログラムの全体}

\lstinputlisting[caption=神経衰弱,label=flipcard]{../src/flipcard.c}

%
\begin{thebibliography}{99}
	\bibitem{1} 田中賢一郎、「ゲームで学ぶ Java Script 入門」、インプレス
\end{thebibliography}
%
\end{document}